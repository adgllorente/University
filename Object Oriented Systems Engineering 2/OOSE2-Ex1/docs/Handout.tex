\documentclass[12pt,a4paper]{article}

\usepackage{a4wide}
\usepackage{citehere}
\usepackage{color}
\usepackage{graphicx}
\usepackage{listings}
\usepackage{subfigure}
\usepackage[basic]{wordlike}
\usepackage{url}
\usepackage{txfonts}


\lstset{language=java}
\definecolor{keywordblue}{rgb}{0.2,0.2,0.8}
\definecolor{commentred}{rgb}{0.8,0.0,0}
\definecolor{stringred}{rgb}{0.5,0,0}
\lstset{keywordstyle=\color{keywordblue}\bf}
\lstset{backgroundcolor=\color{white}}
\lstset{commentstyle=\color{commentred}}
\lstset{stringstyle=\color{stringred}\upshape}
\lstset{tabsize=1}
\lstset{basicstyle=\tt\small}

\title{Weblog Processor}

\author{Tim Storer}

\date{Assessed Exercise 1, Weight 10\%}

\begin{document}

\maketitle

\begin{itemize}
\item Due at 10:00 am on the day after your lab during the week of 15th
  February.
\item Reminder: credit for OOSE2 will be refused unless you submit at
  least one of the two assessed exercises and attend at least three
  OOSE2 lab sessions.
\end{itemize}

This handout describes the first assessed exercise for OOSE2.  Read it
over carefully.  If you do not fully understand what you need to do,
please discuss with your tutor during your lab the week of 1st
February.

You are responsible for modifying an existing, working application to
meet a set of modified or additional requirements.  You have been
provided with the requirements specification and a set of test JUnit
test cases that are used to validate the classes that make up the
application.  You have been given java files for all of the classes
that constitute the application.  Additionally, you have been provided
with the html files generated by javadoc.

\section{Introduction}

The application \lstinline!LogStats! processes log files from web
servers.  For each request received, the web server writes a line in
its log file consisting of a date, in the format \verb!dd/mm/yyyy!,
followed by white space, followed by the domain name of the machine
that submitted the request; a domain name consists of a sequence of
‘.’-separated fields, such as dcs.gla.ac.uk; the last such field is
termed a country code.

For a given period marked by begin and end dates,
\lstinline!LogStats!, keeps track of the number of requests received
from each country code.  For example, assume that the begin date is
01/01/2001 and the end date is 31/12/2001; \lstinline!LogStats! then
processes a log file, keeping only log entries associated with dates
01/01/2001 to 31/12/2001, inclusive.  If the following log file were
processed by \lstinline!LogStats! with these dates,

\

\begin{verbatim}
02/03/2001 a.b.c.d
31/12/2000 g.h.i.d
15/06/2001 x.y.z.q
01/01/2002 x.y.b.q
28/09/2001 r.s.q
\end{verbatim}

\

it would ignore the 2nd and 4th lines (outside of the period); of the
remaining lines, it would indicate that it had seen one log line in
the period associated with country code ``d'', and two log lines
associated with country code ``q''.

Several alterations to the LogStats application are necessary. In each
case, the change should be documented by revising the appropriate test
case (as well as the requirements document) and then be corrected in
the implementation.

\begin{enumerate}
\item The current program does not exactly meet the requirements
  specified in the requirements document: it is required that each
  date be validated (see requirements document).  The current
  implementation of date validation does not meet the requirements.
\item The current program prints out the entries from smallest count
  to highest count; for country codes possessing identical counts, the
  entries are printed out in alphabetic order of country code.  The
  customers are unanimous in demanding that the entries be printed out
  in the reverse order – i.e. from highest count to lowest count; for
  country codes with the same count, you should continue to output
  them in alphabetic order by country code.
\item Finally, customers that have very popular web sites have
  indicated that the program runs EXTREMELY slowly when processing
  large log files with entries from thousands of different machines.
  This is probably due to the way that the country codes are stored in
  a \lstinline!CountryList!.  You need to track down the cause of the
  problem, document it as a test method in
  \lstinline!TimedLogStatsTest! and implement any required
  modifications.
\end{enumerate}

\section{Approach}

The most important first step is to thoroughly read the requirements
document to make sure you understand it.  Then study the java classes
and test cases; it is critical that you understand what each of the
methods are doing AND how they are performing their tasks.  Only when
this is the case are you in a position to make the required
modifications.

\begin{enumerate}
\item First you must understand why the validation that occurs in
  \lstinline!Date! does not meet the requirements. Recall that a leap
  year satisfies the following constraints:

\begin{quotation}
\lstinline!year % 4 == 0!, unless \lstinline!year % 100 == 0!, except if \lstinline!year % 400 == 0!
\end{quotation}
 
for example, 2008 is a leap year, 1900 was not a leap year, 2000 was a
leap year.

\item You must understand how \lstinline!sortByCount()! works in
  \lstinline!CountryList!; you could modify \lstinline!sortByCount()!
  to sort by decreasing count, or you could add a new method that
  performs such a sort by decreasing count.

\item You must understand how the system computation time varies with
  the number of entries processed AND the number of unique country
  codes.  A JUnit test harness, \lstinline!TimedLogStatsTest! , is
  provided to assist you in determining the behaviour or
  \lstinline!LogStats! as a function of the number of entries
  processed and the number of unique country codes.

  The \lstinline!TimedLogStatsTest.testInsertionTiming()!
  method times the amount of time taken for a varying (large) number
  of entries to be inserted into the \lstinline!CountryList!.  This is
  done with the aid of two utility methods:

  \begin{itemize}
  \item \lstinline!generateLargeLogInputStream(nccodes, nentries)!
    returns an input stream representing a large log file.
    \lstinline!nentries! specifies the number of lines in the log
    file, while \lstinline!nccodes! specifies the number of country
    codes to be used.  Invoking
    \lstinline!doLargeCountryInsert(1000,1000)!, for example will
    generate 1000 lines of log entries on the \lstinline!InputStream!,
    each with a unique country code.
  \item \lstinline!doLargeCountryInsert()! produces timing statistics
    on stdout (the Console in Eclipse) for a single insertion run.
  \end{itemize}

  To test the scaling behaviour of \lstinline!LogStats! with the
  number of unique country codes, you need to modify the test case to
  report timings for a range of different log files sizes.  You might
  use \lstinline!doLargeCountryInsert(int,int)! for the following
  values:

  \begin{lstlisting}
    1000 1000
    2000 2000
    4000 4000
    8000 8000
    16000 16000
    32000 32000    
  \end{lstlisting}

  If you generate results like those I describe above, using start
  date of 01/01/1900 and end date of 01/2/2008, you will obtain the
  elapsed time as a function of the number of unique country codes; if
  you plot the elapsed time as a function of the number of unique
  country codes, you should see a pattern emerging.  You must
  determine what is causing this pattern, and design changes to the
  program to eliminate the pattern.  As always, before you change any
  classes, you must modify the appropriate test harnesses.  You should
  also generate a set of results in which the number of unique country
  codes remains constant, but the number of entries in the log file
  vary to see if these give any indication as to the cause of the
  slowdown experienced by customers.
 \end{enumerate}

\section{Set Up}

When you set up Exercise1 using AMS you will obtain an Exercise1
folder.  For this assignment, you will have to switch your Eclipse
workspace to the Exercise1 folder.

Create a new Java project, named OOSE2-Ex1, creating it from the
existing folder Exercise1.  Make sure that the ``create separate
folders for source and class files'' radio button is checked.  At this
point, all of the files in the folder tree descending from Exercise1
will become part of your project in Eclipse.  These files include the
source code and javadoc-generated documentation for the classes and
applications.  The requirements document is available in the /docs
folder.

You are to make your modifications to the files in place.  It is a
good idea to save copies of these original files before you begin to
modify them, just in case you make a mistake.  Systems like CVS or
Subversion are available in Eclipse to support this; if you are not
familiar with such systems, simply making a copy should be sufficient
protection.  I suggest that if you are saving a copy of ``name.ext''
that you name the copy ``name.ext.save''.

Note: do NOT rename any of the {\tt .java} files in {\tt
  Exercise1/src}, any of the {\tt .html} files in {\tt
  Exercise1/docs/api/}, nor any of the {\tt .doc} files in
{\tt Exercise1/docs}.

Open \lstinline!LogStats.java! in Eclipse.  \lstinline!LogStats! is a
console application and takes command arguments.  In order to
familiarise yourself with the \lstinline!LogStats! application, you
can run the application from the console.  Note, that any testing {\em
  must} be performed using the JUnit test case suite.  To run
\lstinline!LogStats! from the console, choose the ``Open Run
Dialog...''  item in the ``Run'' menu; you will see a tab labelled
``(x)= Arguments''.  If you click on that tab, you will see a box
labelled ``Program arguments:''.  There is a sample log file in the
oose2 package.  Type the following into the ``Program arguments:''
box.

\

\begin{verbatim}
01/01/1902 02/02/2008 logs/small.log

\end{verbatim}

\
 
This should produce the following output in the console:

\

\begin{verbatim}
Statistics for oose2/small.log in the date interval [01/01/1902, 02/02/2008]
edu        4
fr         4
it         4
com        8
de         8
uk        20
\end{verbatim}

\

\section{File list in directory Exercise1}
\begin{itemize}
\item {\tt doc/*.doc} – the requirements document
\item {\tt docs/api/} – javadoc files for the classes and
  programs
\item {\tt src/oose2/ex1b/*.java} – source files for the
  classes
\item {\tt src/oose2/ex1b/tests/*.java} – source files for
  the test cases
\item {\tt logs/*.log} log files for testing.
\item {\tt logs/small-2001.01.01-2002.12.31.out} correct output when
  LogStats is invoked on {\tt logs/small.log} with:
  \begin{itemize}
  \item begin date = 01/01/2001
  \item end date = 12/31/2002
  \end{itemize}
\end{itemize}


\section{What you will be submitting}
When you submit your solution to Exercise 1, we will copy the
following files and directories into the AMS system:

\begin{itemize}
\item Exercise1/docs/requirements.doc
\item Exercise1/docs/api/ - API documentation.
\item Exercise1/src/oose2/ex1b/*.java – all classes that have
  been modified
\item Exercise1/src/oose2/ex1b/tests/*.java – all test cases
  that have been modified
\end{itemize}

Any other files that you create anywhere in the Exercise1 tree will
not be copied.

\section{Assessment}

Your submission will be marked on a basis of 10 marks for the
assignment.  Successfully addressing the Date validation bug will be
worth 2 mark, the reverse ordering of the output will be worth 3
marks, and the performance fix will be worth 5 marks.  Note that
addressing the changes requires revision to the accompanying
documentation, as well as both a revised test suite (which you should
do first) and a revised implementation.

\end{document}
